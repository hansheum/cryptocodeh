% Hans's stolen version of Martijn's tikz setup
%Potentially useful resource
%https://nymity.ch/tikz/

\usepackage{pgfpages}
  \usepackage{color}
  \usepackage{colortbl}
  \usepackage{multirow}
  \usepackage{xspace}

  \usepackage{rotating}
  \usepackage{wasysym}
  % \usepackage{marvosym} %% MSM: causes a conflict with \Cross already being defined elsewhere;
                          %%      unclear which symbol relies on marvosym package...

  \usepackage{tikz}
    \usetikzlibrary{calc,scopes,arrows.meta,chains,matrix,positioning,fit,backgrounds,fadings}
    \usetikzlibrary{shapes,shapes.geometric,shapes.symbols,shapes.misc,shapes.callouts}
    \usetikzlibrary{decorations.markings,decorations.pathmorphing,decorations.pathreplacing,decorations.shapes}

    \pgfdeclarelayer{background}
    \pgfdeclarelayer{channel}
    \pgfdeclarelayer{parties}
    \pgfdeclarelayer{timeline}
    \pgfdeclarelayer{reduction}
    \pgfdeclarelayer{foreground}
    % \pgfsetlayers{background,reduction,main,foreground}
    \pgfsetlayers{background,channel,parties,timeline,reduction,main,foreground}

\newcommand{\darkcolor}[2]{{\textcolor{#1!50!black}{#2}}}
  \colorlet{redshaded}{red!25!white}
  \colorlet{shaded}{black!25!white}
  \colorlet{shadedshaded}{black!10!white}
  \colorlet{blackshaded}{black!40!white}

  \colorlet{darkred}{red!80!black}
  \colorlet{darkblue}{blue!80!black}
  \colorlet{darkgreen}{green!80!black}
  \colorlet{darkerred}{red!50!black}
  \colorlet{darkerblue}{blue!50!black}
  \colorlet{darkergreen}{green!50!black}

  \colorlet{washred}{red!20!white}
  \colorlet{washblue}{blue!20!white}
  \colorlet{washgreen}{green!20!white}
  \colorlet{washgray}{gray!20!white}

  \colorlet{drawprim}{darkblue}
  \colorlet{fillprimhi}{white}
  \colorlet{fillprimlo}{blue!20!black!10}

  \colorlet{drawcol}{black!70}
  \colorlet{fillcol}{black!5!blue!5}

  %\colorlet{tightcol}{blue!50!black}
  %\colorlet{loosecol}{red!80!black}
  \colorlet{tightcol}{blue}
  \colorlet{loosecol}{red}

\makeatletter
  \pgfdeclareshape{xor sign}
  {
    \inheritsavedanchors[from=circle] % this is nearly a circle
    \inheritanchorborder[from=circle]
    \inheritanchor[from=circle]{north}
    \inheritanchor[from=circle]{north west}
    \inheritanchor[from=circle]{north east}
    \inheritanchor[from=circle]{center}
    \inheritanchor[from=circle]{west}
    \inheritanchor[from=circle]{east}
    \inheritanchor[from=circle]{mid}
    \inheritanchor[from=circle]{mid west}
    \inheritanchor[from=circle]{mid east}
    \inheritanchor[from=circle]{base}
    \inheritanchor[from=circle]{base west}
    \inheritanchor[from=circle]{base east}
    \inheritanchor[from=circle]{south}
    \inheritanchor[from=circle]{south west}
    \inheritanchor[from=circle]{south east}
    \inheritbackgroundpath[from=circle]
    \foregroundpath{
      \centerpoint%
      \pgf@xc=\pgf@x%
      \pgf@yc=\pgf@y%
      \pgfutil@tempdima=\radius%
      \pgfmathsetlength{\pgf@xb}{\pgfkeysvalueof{/pgf/outer xsep}}%
      \pgfmathsetlength{\pgf@yb}{\pgfkeysvalueof{/pgf/outer ysep}}%
      \ifdim\pgf@xb<\pgf@yb%
        \advance\pgfutil@tempdima by-\pgf@yb%
      \else%
        \advance\pgfutil@tempdima by-\pgf@xb%
      \fi%
      \pgfpathmoveto{\pgfpointadd{\pgfqpoint{\pgf@xc}{\pgf@yc}}{\pgfqpoint{-1\pgfutil@tempdima}{0\pgfutil@tempdima}}}
      \pgfpathlineto{\pgfpointadd{\pgfqpoint{\pgf@xc}{\pgf@yc}}{\pgfqpoint{1\pgfutil@tempdima}{0\pgfutil@tempdima}}}
      \pgfpathmoveto{\pgfpointadd{\pgfqpoint{\pgf@xc}{\pgf@yc}}{\pgfqpoint{0\pgfutil@tempdima}{-1\pgfutil@tempdima}}}
      \pgfpathlineto{\pgfpointadd{\pgfqpoint{\pgf@xc}{\pgf@yc}}{\pgfqpoint{0\pgfutil@tempdima}{1\pgfutil@tempdima}}}
    }
  }

  \pgfdeclareshape{enc box}
  {
    \inheritsavedanchors[from=rectangle] % this is nearly a circle
    \inheritanchorborder[from=rectangle]
    \inheritanchor[from=rectangle]{north}
    \inheritanchor[from=rectangle]{north west}
    \inheritanchor[from=rectangle]{north east}
    \inheritanchor[from=rectangle]{center}
    \inheritanchor[from=rectangle]{west}
    \inheritanchor[from=rectangle]{east}
    \inheritanchor[from=rectangle]{mid}
    \inheritanchor[from=rectangle]{mid west}
    \inheritanchor[from=rectangle]{mid east}
    \inheritanchor[from=rectangle]{base}
    \inheritanchor[from=rectangle]{base west}
    \inheritanchor[from=rectangle]{base east}
    \inheritanchor[from=rectangle]{south}
    \inheritanchor[from=rectangle]{south west}
    \inheritanchor[from=rectangle]{south east}
    \backgroundpath{
      \southwest \pgf@xa=\pgf@x \pgf@ya=\pgf@y%
      \northeast \pgf@xb=\pgf@x \pgf@yb=\pgf@y%
      \pgfpathmoveto{\pgfpoint{\pgf@xa}{\pgf@ya}}
      \pgfpathlineto{\pgfpoint{\pgf@xa}{\pgf@yb}}
      \pgfpathlineto{\pgfpoint{\pgf@xb}{\pgf@yb}}
      \pgfpathlineto{\pgfpoint{\pgf@xb}{\pgf@ya}}
      \pgfpathclose
      \begin{\pgfscope}
      \pgfsetlinewidth{2pt}
      \pgfpathmoveto{\pgfpoint{\pgf@xa}{\pgf@yb}}
      \pgflineto{\pgfpoint{\pgf@xb}{\pgf@yb}}
      \end{\pgfscope}
    }
  }
\makeatother

\tikzfading[name=fade out,
inner color=transparent!0,
outer color=transparent!100]

%%
%% Common typesetting commands
%%
\tikzset{
  ghost/.style={ %creates an invisible element of the correct size
    draw opacity = 0,
    fill opacity = 0,
    text opacity = 0
  },
  ghostly/.style={ %creates an invisible element of the correct size
    draw opacity = 0.2,
    fill opacity = 0.2,
    text opacity = 0
  },
  hv path/.style={to path={-| (\tikztotarget)},rounded corners},
  vh path/.style={to path={|- (\tikztotarget)},rounded corners},
  flowGeneric/.style={
    draw, % draw=black,
    preaction={%under-highlight
      draw,
      white,
      stealth-stealth,
      double=white,
      double distance=2\pgflinewidth,
    },
    rounded corners=2pt,
    nodes={above=-1mm,midway,},
  },
  noflow/.style={   flowGeneric},
  flow/.style={     flowGeneric, >=stealth,->},
  revflow/.style={  flowGeneric, >=stealth,<-},
}

\newcommand{\flow}{\draw[flow]}

%%
%%  Commands for typesetting security games
%%
\tikzset{
  square matrix/.style={
    row sep   = 3mm,
    column sep= 3mm,
  },
  setup/.style={
    rectangle, rounded corners,
    minimum width={1cm},minimum height={1cm},
    draw=white,
    inner color=fillprimhi,
    outer color=fillprimlo,
    align={center},anchor={center},
  },
  redcode/.style={
    rectangle, rounded corners,
    minimum width={1cm},minimum height={1cm},
    draw=white,
    inner color=fillprimhi,
    outer color=fillprimlo,
    align={center},anchor={center},
  },
  challenge/.style={
    rectangle, rounded corners,
    minimum width={1cm},minimum height={1cm},
    draw=drawprim,
    top color=fillprimhi,
    bottom color=fillprimlo,
    align={center},anchor={center},
  },
  power/.style={
    rectangle, rounded corners,
    minimum width={1cm},minimum height={1cm},
    draw=darkergreen,
    top color=fillprimhi,
    bottom color=washgreen,
    align={center},anchor={center},
  },
  group/.style={
    rectangle, minimum width=9mm, minimum height=7mm,
    very thick,
    draw=red!30!black!10,
    fill=yellow!20!red!10,
    anchor=center,
  },
  reduction/.style={
    rectangle, minimum width=9mm, minimum height= 7mm, rounded corners=3mm,
    thick,draw=black,
    top color=fillprimhi,
    bottom color=yellow!20!red!10,
    % font=\ttfamily,
    anchor=center
  },
  adv/.style={
    rectangle, minimum width=9mm, minimum height= 7mm, rounded corners=3mm,
    thick,draw=black,
    top color=washred,
    bottom color=fillprimhi,
    % font=\ttfamily,
    anchor=center
  },
  inout/.style={
    rectangle,anchor=center
  },
  ghost/.style={
    draw opacity = 0,
    fill opacity = 0,
    text opacity = 0
  },
  dline/.style={
    very thick, dashed,
    draw=red!30!black!10,
  },
  tip/.style={->,rounded corners},
}

%%
%%  Commands for typesetting syntax diagrams
%%
\newcommand{\boxheight}{.5cm}
\tikzset{
  alg/.style={
    rectangle,
    align={center},
    inner sep = 0.1cm,
    minimum height= {\boxheight},
    rounded corners=1mm,
    thick,draw=drawprim,
    top color=fillprimhi,
    bottom color=fillprimlo,
    anchor=center,
  },
  io/.style={
    rectangle,anchor=center
  },
  otip/.style={->,rounded corners},
  stip/.style={->,shorten >=1pt,rounded corners},
  spit/.style={<-,shorten <=1pt,rounded corners},
}

%%
%% Commands for typesetting timeline diagrams
%%
\tikzset{
  event/.style={
    circle,
    minimum width={1mm},
    draw=black,
    fill=black,
    align={center},anchor={center},
  },
  timeline/.style={
    rectangle,
    minimum width={1.5mm},
    % minimum height={10mm},
    draw=white,
    fill=white!50!black,
    align={center},anchor={north},
  },
  outro/.style={
    rectangle,
    minimum width={1.5mm},
    minimum height={10mm},
    draw=white,
    top color=white!50!black,
    bottom color=white,
    align={center},anchor={north},
  },
  intro/.style={
    rectangle,
    minimum width={1.5mm},
    minimum height={10mm},
    draw=white,
    top color=white,
    bottom color=white!50!black,
    align={center},anchor={south},
  },
  party/.style={
    rectangle,
    rounded corners = 2mm,
    ultra thick,
    dotted,
    draw=black,
    fill=white,
    anchor=center
  },
  eve/.style={
    rectangle,
    rounded corners = 2mm,
    ultra thick,
    dotted,
    draw=orange,
    anchor=center
  },
  public/.style={
    arrows = {-Stealth[inset=2pt, angle=30:10pt,scale=1.5]},
    orange,
    line width=1.0pt,
    double distance=1.5pt,
  },
  secure/.style={
    arrows = {-Stealth[inset=2pt, angle=30:10pt,scale=2]},
    purple,
    line width=0.5pt,
    double distance=1.5pt,
  },
  % test/.style={
  %   arrows = {-Stealth[inset=2pt, angle=30:10pt,scale=1]},
  %   orange,
  %   line width=0.5pt,
  %   double distance=1.5pt,
  % },
  clear/.style={
    white,
    line width=0.5pt,
    double distance=0.5pt,
  },
}

%%
%% Command for typesetting schemes
%% MSM: might want to integrate enc/dec styles with alg style from syntax?
%%
\tikzset{
  xor/.style={
    shape = xor sign, minimum size=4mm,
    thick,inner sep=0pt,anchor=center,draw=drawcol,
  },
  point/.style={coordinate},
  enc/.style={
    rectangle,
    align={center},
    inner sep = 0.1cm,
    minimum height= {\boxheight},
    rounded corners=1mm,
    thick,draw=drawprim,
    top color=fillprimhi,
    bottom color=fillprimlo,
    anchor=center,
  },
  dec/.style={
    rectangle,
    align={center},
    inner sep = 0.1cm,
    minimum height= {\boxheight},
    rounded corners=1mm,
    thick,draw=drawprim,
    top color=fillprimlo,
    bottom color=fillprimhi,
    % font=\ttfamily,
    anchor=center
  },
  mode/.style={
    rectangle,
    align={center},
    inner sep = 0.1cm,
    minimum height= {\boxheight},
    rounded corners=3mm,
    thick,draw=drawprim,
    top color=fillprimhi,
    bottom color=fillprimlo,
    anchor=center,
  },
  prepost/.style={
    rectangle,
    align={left},
    anchor=west,
  },
}

%% MSM: garbage collection of old tikz commands...
%% WARNING: stealth is deprecated, use Stealth instead!
\iffalse
  \tikzset{
    mode matrix/.style={
      matrix of nodes,
      nodes in empty cells,
      row sep=2mm,
      column sep=5mm,
      align={center},
      text centered,
      inner sep=0.1cm,
      % draw=black,
      % rectangle,
    },
    box/.style={
      align={center},
      inner sep=0.1cm,
      minimum height={\boxheight},
      draw=black,
      rectangle,
      %rounded corners,
    },
    diagram matrix/.style={
    	row sep   = 2mm,
    	column sep= 5mm,
    },
    square matrix/.style={
      row sep   = 3mm,
      column sep= 3mm,
    },
    %
    flowGeneric/.style={
      draw=black,
      preaction={%under-highlight
        draw,
        white,
        stealth-stealth,
        double=white,
        double distance=2\pgflinewidth,
      },
      rounded corners=2pt,
      nodes={above=-1mm,midway,},
    },
    flow/.style={     flowGeneric, >=stealth,->},
    revflow/.style={  flowGeneric, >=stealth,<-},
    possible/.style={
      flow, densely dotted},
    lgap/.style={ xshift=-8mm },
    rgap/.style={ xshift=+3mm },
    %
    XOR/.style={
      draw,
      fill=white,
      minimum size=13pt,
      circle,
      append after command={
        [shorten >=\pgflinewidth, shorten <=\pgflinewidth,]
        (\tikzlastnode.north) edge (\tikzlastnode.south)
        (\tikzlastnode.east) edge (\tikzlastnode.west)
      }
    },
    Fbox/.style={box},
    diagram node/.style={
      draw={black}, rectangle, rounded corners,
      minimum width={1cm},minimum height={1cm},
      fill={none}, fill opacity={0.2}, text opacity={1},
      align={center},anchor={center},
    },
    onslide/.code args={<#1>#2}{%
    \only<#1>{\pgfkeysalso{#2}} % \pgfkeysalso doesn't change the path
    },
  }

  \tikzset{
    challenge/.style={
      rectangle, rounded corners,
      minimum width={1cm},minimum height={1cm},
      draw=drawprim,
      top color=fillprimhi,
      bottom color=fillprimlo,
      align={center},anchor={center},
    },
    power/.style={
      rectangle, rounded corners,
      minimum width={1cm},minimum height={1cm},
      draw=darkergreen,
      top color=fillprimhi,
      bottom color=washgreen,
      align={center},anchor={center},
    },
    group/.style={
      rectangle, minimum width=9mm, minimum height=7mm,
      very thick,
      draw=red!30!black!10,
      fill=yellow!20!red!10,
      anchor=center,
    },
    hidden/.style={
      rectangle,
      fill=white,
      opacity=0.8,
    },
    highlight/.style={
      rectangle,
      fill = yellow,
      path fading=fade out,
    },
    cluster/.style={
      rectangle,
      fill = green,
      path fading=fade out,
    },
    whiten/.style={
      rectangle,
      fill=white,
      opacity=1.0,
    },
    dline/.style={
      very thick, dashed,
      draw=red!30!black!10,
    },
    func/.style={
      rectangle,
      align={center},
      inner sep = 0.1cm,
      minimum height= {\boxheight},
      rounded corners=1mm,
      thick,draw=drawprim,
      top color=fillprimhi,
      bottom color=fillprimlo,
      anchor=center,
    },
    enc/.style={
      rectangle,
      align={center},
      inner sep = 0.1cm,
      minimum height= {\boxheight},
      rounded corners=1mm,
      thick,draw=drawprim,
      top color=fillprimhi,
      bottom color=fillprimlo,
      anchor=center,
    },
    emptyfunc/.style={
      rectangle, minimum width=9mm, minimum height=7mm,
      thick,
      draw=drawprim,
      anchor=center
    },
    blackbox/.style={
      rectangle,aspect=1, ultra thick, draw=black, fill=black!30, anchor=center
    },
    sbox/.style={
      rectangle,aspect=1, thick, draw=black, anchor=center, fill=washii
    },
    pbox/.style={
      rectangle,aspect=1, thick, draw=black, anchor=center, fill=white
    },
    pibox/.style={
      rectangle,aspect=1, thick, draw=black, anchor=center, fill=orange!20!white
    },
    rbox/.style={
      rectangle, rounded corners=3mm, aspect=1, ultra thick, draw=green!50!black, anchor=center, fill=green!10!white
    },
    fbox/.style={
      rectangle, rounded corners=2mm, aspect=1, thick, draw=black, anchor=center, fill=blue!30!white
    },
    oracle/.style= {
      cloud, cloud ignores aspect, cloud puffs=11, draw=green, thick, fill=green!20, text width=18mm, text centered
    },
    codacle/.style= {
      tape, draw=green, thick, fill=green!20, text width=18mm, text centered
    },
    % MSM: repurposed party
    % party/.style={
    %   rectangle,aspect=1, ultra thick, draw=black, fill=green!30, anchor=center
    % },
    partyin/.style={cloud callout, cloud puffs=15, aspect=2.5, cloud puff arc=120, shading=ball,text=white,callout relative pointer={#1}
    },
    partyout/.style={ellipse callout, aspect=2.5, shading=ball,text=white,callout relative pointer={#1}
    },
    dec/.style={
      rectangle,minimum width=7mm, minimum height= 7mm,% rounded corners=3mm,
      thick,draw=drawprim,
      top color=fillprimlo,
      bottom color=fillprimhi,
      % font=\ttfamily,
      anchor=center
    },
    enc2/.style={
      rectangle,minimum width=7mm, minimum height= 7mm,% rounded corners=3mm,
      thick,draw=green!80!white,
      top color=green!80!white,
      bottom color=green!80!white,
      % font=\ttfamily,
      anchor=center
    },
    denc/.style={
      rectangle,minimum width=14mm, minimum height= 7mm,% rounded corners=3mm,
      thick,draw=drawprim,
      top color=fillprimhi,
      bottom color=fillprimlo,
      % font=\ttfamily,
      anchor=center
    },
    adv/.style={
      rectangle,minimum width=9mm, minimum height= 7mm, rounded corners=3mm,
      thick,draw=black,
      top color=washred,
      bottom color=fillprimhi,
      % font=\ttfamily,
      anchor=center
    },
    proc/.style={
      rectangle,minimum width=9mm, minimum height= 7mm, rounded corners=3mm,
      thick,draw=drawprim,
      anchor=center
    },
    list/.style={
      rectangle,minimum width=2mm, minimum height= 7mm, rounded corners=1mm,
      thick,draw=drawprim,
      anchor=center
    },
    merlist/.style={
      rectangle,minimum width=3mm, minimum height= 9mm, rounded corners=1mm,
      thick,draw=drawprim,
      anchor=center
    },
    inout/.style={
      rectangle,anchor=center
    },
    pinout/.style={
      rectangle, minimum height = 6mm,
      text height = 5mm, text depth = 2pt,
      anchor=base
    },
    abit/.style={
      rectangle, anchor=center
    },
    ctxt/.style={
      rectangle, minimum height= 5 mm, rounded corners = 1mm,
      thick, draw=darkblue,
      top color = white,
      bottom color = washblue,
      anchor=center
    },
    fake/.style={
      rectangle, minimum height= 5 mm, rounded corners = 1mm,
      thick, draw=darkred,
      top color = washred,
      bottom color = white,
      anchor=center
    },
    okfrag/.style={
      rectangle, minimum height= 9 mm, rounded corners = 3mm,
      thick, draw=darkgreen,
      fill opacity = 0.5,
      top color = washgreen,
      bottom color = white,
      anchor=center
    },
    badfrag/.style={
      rectangle, minimum height= 7 mm, rounded corners = 3mm,
      thick, draw=darkred,
      fill opacity = 0.5,
      top color = washred,
      bottom color = white,
      anchor=center
    },
    minixor/.style={
      shape = xor sign, minimum size=2mm,
      thick,inner sep=0pt,anchor=center,draw=drawcol,
    },
    minienc/.style={
      rectangle,minimum width=3mm, minimum height= 3mm,
      thick,draw=drawprim,
      top color=fillprimhi,
      bottom color=fillprimlo,
      % font=\ttfamily,
      anchor=center
    },
    xor/.style={
      shape = xor sign, minimum size=4mm,
      thick,inner sep=0pt,anchor=center,draw=drawcol,
    },
    hash/.style={
      trapezium, minimum size = 15 mm,
      trapezium left angle=90, trapezium right angle = -60, shape border rotate = 90,
      thick,draw=drawcol,
      fill=fillcol
    },
    hash1/.style={
      trapezium, minimum size = 15 mm,
      trapezium left angle=90, trapezium right angle = -70, shape border rotate = 90,
      thick,draw=drawcol,
      fill=fillcol
    },
    hash2/.style={
      trapezium, minimum size = 15 mm,
      trapezium left angle=90, trapezium right angle = -80, shape border rotate = 90,
      thick,draw=drawcol,
      fill=fillcol
    },
    chop/.style={
      trapezium,
      trapezium left angle=90, trapezium right angle = -70, shape border rotate = 90,
      thick,draw=drawcol,
      fill=fillcol
    },
    altchop/.style={
      trapezium,
      trapezium left angle=90, trapezium right angle = -110, shape border rotate = 90,
      thick,draw=drawcol,
      fill=fillcol
    },
    tick/.style={
      strike out,
      minimum height = 4pt,
      minimum width = 2pt,
      inner sep = 0pt,
      thick, draw=drawcol,
      -
    },
    tn/.style={
       font = \footnotesize,
       text = drawcol
    },
    skip loop/.style={
      to path={-- ++(0,#1) -| (\tikztotarget)}
    },
    point/.style={coordinate},
    >=stealth',
    tip/.style={->,rounded corners},
    tipsiz/.style={-,rounded corners},
    stip/.style={->,shorten >=1pt,rounded corners},
    dstip/.style={->,double,shorten >=1pt,rounded corners},
    pit/.style={<-,rounded corners},
    spit/.style={<-,shorten <=1pt,rounded corners},
    sii/.style={<->,shorten <=1pt,shorten >=1pt,rounded corners},
    hv path/.style={to path={-| (\tikztotarget)},rounded corners},
    vh path/.style={to path={|- (\tikztotarget)},rounded corners},
    label/.style={
        font = \footnotesize,
        text = drawcol
        , postaction={decorate, decoration={markings, mark = at position 4mm with \node #1;}}
    }
  }
\fi

%% MSM: culled down version of Guy's tikz box-commands
\iffalse
  \newcommand{\figIO}[1]{|[inout]| \ensuremath{#1}}
  \newcommand{\figF}{|[func]| \ensuremath{\mathcal{F}_{k_F}}}
  \newcommand{\figFk}[1]{|[func]| \ensuremath{\mathcal{F}_{#1}}}
  \newcommand{\figUk}[1]{|[func]| \ensuremath{\mathcal{U}_{#1}}}
  % \newcommand{\figFk}[1][]{|[func]| \ensuremath{\mathcal{F}_{\if!#1!{k_F}\else{#1}\fi}}}
  \newcommand{\figFe}{|[func]| \ensuremath{\mathcal{F}_{k}}}
  \newcommand{\figivE}{|[enc]| \ensuremath{\mathrm{iv}\mathcal{E}_{k_E}}}
  \newcommand{\figivD}{|[enc]| \ensuremath{\mathrm{iv}\mathcal{D}_{k_E}}}
  \newcommand{\figE}{|[enc]| \ensuremath{\mathcal{E}_{k_E}}}
  \newcommand{\figD}{|[enc]| \ensuremath{\mathcal{D}_{k_E}}}
  \newcommand{\figT}{|[func]| \ensuremath{\mathcal{T}_{k_M}}}
  \newcommand{\figV}{|[func]| \ensuremath{\mathcal{V}_{k_M}}}

  \newcommand{\figEr}{|[box]| \ensuremath{\mathrm{E}_{k}}}
  \newcommand{\figXOR}{|[inner sep=0]|\ensuremath{\bigoplus}}

  \newcommand{\figMath}[1]{|[func]| \ensuremath{#1}}

  \newcommand*{\StrikeThruDistance}{0.15cm}%
  \newcommand{\diagramgap}{1mm}
  \newcommand{\boxheight}{.5cm}
  \newcommand{\dstep}{1}
  \newcommand{\rstep}{1}
\fi
